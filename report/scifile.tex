% Use only LaTeX2e, calling the article.cls class and 12-point type.

\documentclass[12pt]{article}

\usepackage{scicite}
\usepackage[utf8]{inputenc} %utf8 % lettere accentate da tastiera
\usepackage[english]{babel} % lingua del documento
\usepackage[T1]{fontenc} % codifica dei font
\usepackage[backend=bibtex,sorting=none, backref=true]{biblatex}
\usepackage{times}
\usepackage[hidelinks]{hyperref}
\usepackage{csquotes}
\usepackage{amssymb}
\usepackage{amsmath}
\usepackage{graphicx}
\usepackage{subfigure}
\usepackage{caption}
\usepackage{mathtools}
\usepackage{listings,lstautogobble}
\usepackage{xcolor}
\usepackage{float}

\definecolor{gray}{gray}{0.5}
\colorlet{commentcolour}{green!50!black}

\colorlet{stringcolour}{red!60!black}
\colorlet{keywordcolour}{blue}
\colorlet{exceptioncolour}{yellow!50!red}
\colorlet{commandcolour}{magenta!90!black}
\colorlet{numpycolour}{blue!60!green}
\colorlet{literatecolour}{magenta!90!black}
\colorlet{promptcolour}{green!50!black}
\colorlet{specmethodcolour}{violet}

\newcommand*{\framemargin}{3ex}

\newcommand*{\literatecolour}{\textcolor{literatecolour}}

\newcommand*{\pythonprompt}{\textcolor{promptcolour}{{>}{>}{>}}}

\lstdefinestyle{python}{
	language=python,
	showtabs=true,
	tab=,
	tabsize=4,
	basicstyle=\ttfamily\footnotesize,
	stringstyle=\color{stringcolour},
	showstringspaces=false,
	keywordstyle=\color{keywordcolour}\bfseries,
	emph={as,and,break,class,continue,def,yield,del,elif ,else,%
		except,exec,finally,for,from,global,if,in,%
		lambda,not,or,pass,print,raise,return,try,while,assert,with},
	emphstyle=\color{blue}\bfseries,
	emph={[2]True, False, None},
	emphstyle=[3]\color{commandcolour},
	morecomment=[s]{"""}{"""},
	commentstyle=\color{commentcolour}\slshape,
	emph={array, matmul, transpose, float32},
	emphstyle=[4]\color{numpycolour},
	emph={[5]assert,yield},
	emphstyle=[5]\color{keywordcolour}\bfseries,
	emph={[6]range},
	emphstyle={[6]\color{keywordcolour}\bfseries},
	literate=*%
	{:}{{\literatecolour:}}{1}%
	{=}{{\literatecolour=}}{1}%
	{-}{{\literatecolour-}}{1}%
	{+}{{\literatecolour+}}{1}%
	{*}{{\literatecolour*}}{1}%
	{**}{{\literatecolour{**}}}2%
	{/}{{\literatecolour/}}{1}%
	{//}{{\literatecolour{//}}}2%
	{!}{{\literatecolour!}}{1}%
	{<}{{\literatecolour<}}{1}%
	{>}{{\literatecolour>}}{1}%
	{>>>}{\pythonprompt}{3},
	frame=trbl,
	rulecolor=\color{black!40},
	backgroundcolor=\color{gray!5},
	breakindent=.5\textwidth,
	frame=single,
	breaklines=true,
	basicstyle=\ttfamily\footnotesize,%
	keywordstyle=\color{keywordcolour},%
	emphstyle={[7]\color{keywordcolour}},%
	emphstyle=\color{exceptioncolour},%
	literate=*%
	{:}{{\literatecolour:}}{2}%
	{=}{{\literatecolour=}}{2}%
	{-}{{\literatecolour-}}{2}%
	{+}{{\literatecolour+}}{2}%
	{*}{{\literatecolour*}}2%
	{**}{{\literatecolour{**}}}3%
	{/}{{\literatecolour/}}{2}%
	{//}{{\literatecolour{//}}}{2}%
	{!}{{\literatecolour!}}{2}%
	{<}{{\literatecolour<}}{2}%
	{<=}{{\literatecolour{<=}}}3%
	{>}{{\literatecolour>}}{2}%
	{>=}{{\literatecolour{>=}}}3%
	{==}{{\literatecolour{==}}}3%
	{!=}{{\literatecolour{!=}}}3%
	{+=}{{\literatecolour{+=}}}3%
	{-=}{{\literatecolour{-=}}}3%
	{*=}{{\literatecolour{*=}}}3%
	{/=}{{\literatecolour{/=}}}3%
}

\lstnewenvironment{python}
{\lstset{style=python}}
{}

\topmargin 0.0cm
\oddsidemargin 0.2cm
\textwidth 16cm 
\textheight 21cm
\footskip 1.0cm

\newenvironment{sciabstract}{%
	\begin{quote} \bf}
	{\end{quote}}


\renewcommand\refname{References and Notes}

\newcounter{problem}
\newcounter{solution}

\newcommand\Problem{%
	\stepcounter{problem}%
	\textbf{\theproblem.}~%
	\setcounter{solution}{0}%
}

\newcommand\Solution{%
	\textbf{Solution:}\\%
}

\newcommand\ASolution{%
	\stepcounter{solution}%
	\textbf{Solution \solution:}\\%
}

\parindent 0in
\parskip 1em

\newcounter{lastnote}
\newenvironment{scilastnote}{%
	\setcounter{lastnote}{\value{enumiv}}%
	\addtocounter{lastnote}{+1}%
	\begin{list}%
		{\arabic{lastnote}.}
		{\setlength{\leftmargin}{.22in}}
		{\setlength{\labelsep}{.5em}}}
	{\end{list}}

\title{Deep Learning Lab \\ \Large{Assignment 1: Neural Networks} \\[0.3em] \normalsize{Faculty of Informatics} \\ \normalsize{Università della Svizzera Italiana}}


\author {{Giorgia Adorni}	\\ \normalsize{giorgia.adorni@usi.ch}}


\date{\today}

\bibliography{scibib}

%%%%%%%%%%%%%%%%% END OF PREAMBLE %%%%%%%%%%%%%%%%

\begin{document} 
	
	% Double-space the manuscript.
	%\baselineskip24pt
	
	\maketitle 
	
	Consider the polynomial $p$ given by
	\begin{equation*}
	 p(x)=x^3+2x^2-4x-8=\sum_{i=1}^4 w_i^*x^{i-1} \mbox{,}
	\end{equation*} 
	
	where $\textbf{w}^*=[-8,-4,2,1]^T$.
	
	Consider also an iid dataset $\mathcal{D} = \{(x_i, y_i)\}^N_{i=1}$, where $y_i = p(x_i)+\epsilon_i$, and each $\epsilon_i$ is drawn from a normal distribution with mean zero and standard deviation $\sigma = \frac{1}{2}$.
	
	If the vector $\textbf{w}^*$ were unknown, linear regression could estimate it given the dataset $\mathcal{D}$. This would require applying a feature map to transform the
	original dataset $\mathcal{D}$ into an expanded dataset $\mathcal{D}'= \{(x_i, y_i)\}^N_{i=1}$ , where $x_i = [1,x_i,x_i^2,x_i^3]$.
	
	%%%%%
	\section{Introduction}
	The scope of this {project} is to perform polynomial regression using a dataset $\mathcal{D}'$, in particular finding an estimate of $\textbf{w}^*=[-8,-4,2,1]^T$ supposing that such vector is unknown.\\
	An interval $[-3, 2]$ for $x_i$, a sample of size $100$ created with a seed of $0$ for training, and a sample of size $100$ created with a seed of $1$ for validation, and $\sigma = \frac{1}{2}$ were assumed.
	
	\section{Tuning the learning rate}
	The learning rate is a configurable hyper-parameter that represent the positive amount of which weights are updated during the training of a neural network.\\
	I tried to discover a suitable learning rate via trial, setting the initial number of iterations to $2000$.
	In the first test I set the learning rate to a traditional default value of $0.1$. In this case, the value is too high that the algorithm diverges.\\
	Hence, I choose to decrease the value to $0.01$ obtaining a validation loss of $0.22$, that is a good results for the gradient descent.
		
	\section{Iterations and early stopping}
	Fixed the learning rate, I decided to use the \textbf{early stopping} rule in order to abort the training procedure when the performance on the validation set stops to improving and therefore avoid over-fitting. \\
	In particular, I measured the validation loss in each iteration, keeping track of the lowest one, and I stopped the training when the validation loss has not improved, compared to the best, after $10$ steps.\\
	In this examining case, after $1248$ iterations the model reach the best loss of $0.2177$.
		\centering
		\includegraphics[width=1\linewidth]{../src/img/model1-dataset.png}
		\captionof{figure}{model1-dataset}
		\label{fig:model1-dataset}
		\end{minipage}
		~
		\begin{minipage}[c]{.5\textwidth}
			\centering
			\includegraphics[width=1\linewidth]{../src/img/model1-polynomial.png}
			\captionof{figure}{model1-polynomial}
			\label{fig:model1-polynomial}
		\end{minipage}
	\end{figure}

		\begin{figure}
		\begin{minipage}[c]{.5\textwidth}
			\centering
			\includegraphics[width=1\linewidth]{../src/img/model2-dataset.png}
			\captionof{figure}{model2-dataset}
			\label{fig:model2-dataset}
		\end{minipage}
		~
		\begin{minipage}[c]{.5\textwidth}
			\centering
			\includegraphics[width=1\linewidth]{../src/img/model2-polynomial.png}
			\captionof{figure}{model2-polynomial}
			\label{fig:model2-polynomial}
		\end{minipage}
	\end{figure}

		\begin{figure}
		\begin{minipage}[c]{.5\textwidth}
			\centering
			\includegraphics[width=1\linewidth]{../src/img/model3-dataset.png}
			\captionof{figure}{model3-dataset}
			\label{fig:model3-dataset}
		\end{minipage}
		~
		\begin{minipage}[c]{.5\textwidth}
			\centering
			\includegraphics[width=1\linewidth]{../src/img/model3-polynomial.png}
			\captionof{figure}{model3-polynomial}
			\label{fig:model3-polynomial}
		\end{minipage}
	\end{figure}

		\begin{figure}
		\begin{minipage}[c]{.5\textwidth}
			\centering
			\includegraphics[width=1\linewidth]{../src/img/model4-dataset.png}
			\captionof{figure}{model4-dataset}
			\label{fig:model4-dataset}
		\end{minipage}
		~
		\begin{minipage}[c]{.5\textwidth}
			\centering
			\includegraphics[width=1\linewidth]{../src/img/model4-polynomial.png}
			\captionof{figure}{model4-polynomial}
			\label{fig:model4-polynomial}
		\end{minipage}
	\end{figure}

		\begin{figure}
		\begin{minipage}[c]{.5\textwidth}
			\centering
			\includegraphics[width=1\linewidth]{../src/img/model5-dataset.png}
			\captionof{figure}{model5-dataset}
			\label{fig:model5-dataset}
		\end{minipage}
		~
		\begin{minipage}[c]{.5\textwidth}
			\centering
			\includegraphics[width=1\linewidth]{../src/img/model5-polynomial.png}
			\captionof{figure}{model5-polynomial}
			\label{fig:model5-polynomial}
		\end{minipage}
	\end{figure}

		\begin{figure}
		\begin{minipage}[c]{.5\textwidth}
			\centering
			\includegraphics[width=1\linewidth]{../src/img/model6-dataset.png}
			\captionof{figure}{model6-dataset}
			\label{fig:model6-dataset}
		\end{minipage}
		~
		\begin{minipage}[c]{.5\textwidth}
			\centering
			\includegraphics[width=1\linewidth]{../src/img/model6-polynomial.png}
			\captionof{figure}{model6-polynomial}
			\label{fig:model6-polynomial}
		\end{minipage}
	\end{figure}

		\begin{figure}
		\begin{minipage}[c]{.5\textwidth}
			\centering
			\includegraphics[width=1\linewidth]{../src/img/model7-dataset.png}
			\captionof{figure}{model7-dataset}
			\label{fig:model7-dataset}
		\end{minipage}
		~
		\begin{minipage}[c]{.5\textwidth}
			\centering
			\includegraphics[width=1\linewidth]{../src/img/model7-polynomial.png}
			\captionof{figure}{model7-polynomial}
			\label{fig:model7-polynomial}
		\end{minipage}
	\end{figure}

	\begin{figure}
		\begin{minipage}[c]{.5\textwidth}
			\centering
			\includegraphics[width=1\linewidth]{../src/img/model8-dataset.png}
			\captionof{figure}{model8-dataset}
			\label{fig:model8-dataset}
		\end{minipage}
		~
		\begin{minipage}[c]{.5\textwidth}
			\centering
			\includegraphics[width=1\linewidth]{../src/img/model8-polynomial.png}
			\captionof{figure}{model8-polynomial}
			\label{fig:model8-polynomial}
		\end{minipage}
	\end{figure}

	
\end{document}




















