\documentclass[a4paper,12pt]{article} % This defines the style of your paper

\usepackage[top = 2.5cm, bottom = 2.5cm, left = 2.5cm, right = 2.5cm]{geometry} 
\usepackage[utf8]{inputenc} %utf8 % lettere accentate da tastiera
\usepackage[english]{babel} % lingua del documento
\usepackage[T1]{fontenc} % codifica dei font

\usepackage{multirow} % Multirow is for tables with multiple rows within one 
%cell.
\usepackage{booktabs} % For even nicer tables.

\usepackage{graphicx} 

\usepackage{setspace}
\setlength{\parindent}{0in}

\usepackage{float}

\usepackage{fancyhdr}

\usepackage{caption}
\usepackage{amssymb}
\usepackage{amsmath}
\usepackage{mathtools}

\usepackage[hidelinks]{hyperref}
\usepackage{csquotes}
\usepackage{subfigure}

\pagestyle{fancy}

\fancyhf{}

\lhead{\footnotesize Deep Learning Lab: Assigment 1}

\rhead{\footnotesize Giorgia Adorni}

\cfoot{\footnotesize \thepage} 

\begin{document}
	

	\thispagestyle{empty}  
	
	\begin{tabular}{p{15.5cm}} 
		{\large \bf Deep Learning Lab} \\
		Università della Svizzera Italiana \\ Faculty of Informatics \\ \today  \\
		\hline
		\\
	\end{tabular} 
	
	\vspace*{0.3cm} 
	
	\begin{center}
		{\Large \bf Assignment 1: Neural Networks}
		\vspace{2mm}
		
		{\bf Giorgia Adorni (giorgia.adorni@usi.ch)}
		
	\end{center}  
	
	\vspace{0.4cm}
	
	%%%%%%%%%%%%%%%%%%%%%%%%%%%%%%%%%%%%%%%%%%%%%%%%
	%%%%%%%%%%%%%%%%%%%%%%%%%%%%%%%%%%%%%%%%%%%%%%%%
	
	% Up until this point you only have to make minor changes for every week 
	%(Number of the homework). Your write up 
	%essentially starts here.
	
	Consider the polynomial $p$ given by
	\begin{equation*}
	p(x)=x^3+2x^2-4x-8=\sum_{i=1}^4 w_i^*x^{i-1} \mbox{,}
	\end{equation*} 
	
	where $\textbf{w}^*=[-8,-4,2,1]^T$.
	
	Consider also an iid dataset $\mathcal{D} = \{(x_i, y_i)\}^N_{i=1}$, where $y_i = p(x_i)+\epsilon_i$, and each $\epsilon_i$ is drawn from a normal distribution with mean zero and standard deviation $\sigma = \frac{1}{2}$.
	
	If the vector $\textbf{w}^*$ were unknown, linear regression could estimate it given the dataset $\mathcal{D}$. This would require applying a feature map to transform the
	original dataset $\mathcal{D}$ into an expanded dataset $\mathcal{D}'= \{(x_i, y_i)\}^N_{i=1}$ , where $x_i = [1,x_i,x_i^2,x_i^3]$.
	
	%%%%%
	\section{Introduction}
	The scope of this {project} is to perform polynomial regression using a dataset $\mathcal{D}'$, in particular finding an estimate of $\textbf{w}^*=[-8,-4,2,1]^T$ supposing that such vector is unknown.\\
	An interval $[-3, 2]$ for $x_i$, a sample of size $100$ created with a seed of $0$ for training, and a sample of size $100$ created with a seed of $1$ for validation, and $\sigma = \frac{1}{2}$ were assumed.
	
	\section{Tuning the Learning Rate}
	The learning rate is a configurable hyper-parameter that represent the positive amount of which weights are updated during the training of a neural network.\\
	I tried to discover a suitable learning rate via trial, setting the initial 
	number of iterations to $2000$, that is a high value for the quantity of 
	parameters present.
	In the first test I set the learning rate to a traditional default value of $0.1$. In this case, the value is too high that the algorithm diverges.\\
	Hence, I choose to decrease the value to $0.01$ obtaining a validation loss of $0.22$, that is a good results for the gradient descent.
	
	\section{Iterations and Early Stopping}
	Fixed the learning rate, I decided to reduce the number of iterations using 
	the \textit{early stopping}. This rule is able to abort the training 
	procedure when the performance on the validation set stops to improving and 
	therefore avoid over-fitting. \\
	In particular, I measured the validation loss in each iteration, keeping track of the lowest one, and I stopped the training when the validation loss has not improved, compared to the best, after $10$ steps.\\
	In this examining case, after $1248$ iterations the model reach the best loss of $0.2177$.
	
	\section{Loss}
	%FIXME
	I used {TensorBoard} to display the loss curve as function of the gradient 
	descent iterations, for both the 
	training and validation set (over time (iterations)), which are shown in 
	the Figure \ref{fig:model1-loss}.
	
	\begin{figure}[ht]
		\centering
		\includegraphics[width=0.5\linewidth]{../src/img/loss/model1-es-loss.jpg}
		\captionof{figure}{model1-loss}
		\label{fig:model1-loss}
	\end{figure}
	
	The two training and validation curves start at $29.69$ and $29.10$ respectively, and go down to $0.24$ and $0.22$ at the $1248\mathrm{th}$ iteration.
	It is therefore seen that the validation loss is lower than the training loss. The reason why this happens is that the training loss is measured during each epoch while validation loss is computed at the end of the learning phase of the same epoch.
	
	\section{Dataset Generation and Polynomial Regression} 
	Considered the polynomial defined by $\textbf{w}^*=[-8,-4,2,1]^T$, Figure 
	\ref{fig:model1-dataset} visualises the training and the validation 
	datasets generated, while Figure \ref{fig:model1-polynomial} shows 
	the polynomial defined by $\textbf{w}^*$ and the polynomial estimated 
	$\hat{\textbf{w}}$.
	
	\begin{figure}[H]
		\begin{minipage}[c]{.5\textwidth}
			\centering
			\includegraphics[width=1\linewidth]{../src/img/model1-dataset.png}
			\captionof{figure}{model1-dataset}
			\label{fig:model1-dataset}
		\end{minipage}
		~
		\begin{minipage}[c]{.5\textwidth}
			\centering
			\includegraphics[width=1\linewidth]{../src/img/model1-polynomial.png}
			\captionof{figure}{model1-polynomial}
			\label{fig:model1-polynomial}
		\end{minipage}
	\end{figure}
	
 	The actual polynomial $\textbf{w}^*=[-8,-4,2,1]^T$ and the estimated one 	
 	$\hat{\textbf{w}}=[-7.86,-4.08,1.98,1.01]^T$ are very close.
 	From the plot in Figure \ref{fig:model1-polynomial+dataset} it is clear 
 	that the estimated polynomial curve is able to fit well enough the data.
 	
	\begin{figure}[H]
		\centering
		\includegraphics[width=0.8\linewidth]{../src/img/model1-polynomial+dataset.png}
		\captionof{figure}{model1-polynomial+dataset}
		\label{fig:model1-polynomial+dataset}
	\end{figure}
	
	\section{Training Set Reduction}
	From the below pictures we can observe that reducing the training dataset 
	from $100$ to $50$, $10$, and finally $5$ observations, the model is not 
	able to understand the pattern in the set. In fact, observing the loss 
	curves, while with $50$ observations the model continuing performs well, by 
	reducing the number of samples down to $5$ the algorithm diverges.
	
	\begin{figure}[H]
		\begin{minipage}[t]{.5\textwidth}
			\centering
			\includegraphics[width=0.85\linewidth]{../src/img/loss/model2-loss.jpg}
			\captionof{figure}{model2-loss}
			\label{fig:model2-loss}
		\end{minipage}
		~
		\begin{minipage}[t]{.5\textwidth}
			\centering
			\includegraphics[width=1\linewidth]{../src/img/model2-polynomial+dataset.png}
			\captionof{figure}{model2-polynomial+dataset}
			\label{fig:model2-polynomial+dataset}
		\end{minipage}
	\caption{Training set 50}
	\end{figure}
	
	\begin{figure}[H]
		\begin{minipage}[t]{.5\textwidth}
			\centering
			\includegraphics[width=0.85\linewidth]{../src/img/loss/model3-loss.jpg}
			\captionof{figure}{model3-loss}
			\label{fig:model3-loss}
		\end{minipage}
		~
		\begin{minipage}[t]{.5\textwidth}
			\centering
			\includegraphics[width=1\linewidth]{../src/img/model3-polynomial+dataset.png}
			\captionof{figure}{model3-polynomial+dataset}
			\label{fig:model3-polynomial+dataset}
		\end{minipage}
	\end{figure}
	
	\begin{figure}[H]
		\begin{minipage}[t]{.5\textwidth}
			\centering
			\includegraphics[width=0.85\linewidth]{../src/img/loss/model4-loss.jpg}
			\captionof{figure}{model4-loss}
			\label{fig:model4-loss}
		\end{minipage}
		~
		\begin{minipage}[t]{.5\textwidth}
			\centering
			\includegraphics[width=1\linewidth]{../src/img/model4-polynomial+dataset.png}
			\captionof{figure}{model4-polynomial+dataset}
			\label{fig:model4-polynomial+dataset}
		\end{minipage}
	\end{figure}
	
	\section{Sigma Increase}
	From the below pictures we can observe that if we further increase the  
	standard deviation, during the creation of the dataset, from $0.5$ to $2$, 
	$4$, and finally $8$, the model performances worsen a lot. In fact, the 
	losses increase respectively to $3.48$, $13.93$ and $55.74$. Moreover, the 
	gap between the training and the validation loss growth.
	One solution to improve the performance could be to increase the 
	observations in the training set.
	
	\begin{figure}[H]
		\begin{minipage}[t]{.5\textwidth}
			\centering
			\includegraphics[width=0.85\linewidth]{../src/img/loss/model5-loss.jpg}
			\captionof{figure}{model5-loss}
			\label{fig:model5-loss}
		\end{minipage}
		~
		\begin{minipage}[t]{.5\textwidth}
			\centering
			\includegraphics[width=1\linewidth]{../src/img/model5-polynomial+dataset.png}
			\captionof{figure}{model5-polynomial+dataset}
			\label{fig:model5-polynomial+dataset}
		\end{minipage}
	\end{figure}
	
	\begin{figure}[H]
		\begin{minipage}[t]{.5\textwidth}
			\centering
			\includegraphics[width=0.85\linewidth]{../src/img/loss/model6-loss.jpg}
			\captionof{figure}{model6-loss}
			\label{fig:model6-loss}
		\end{minipage}
		~
		\begin{minipage}[t]{.5\textwidth}
			\centering
			\includegraphics[width=1\linewidth]{../src/img/model6-polynomial+dataset.png}
			\captionof{figure}{model6-polynomial+dataset}
			\label{fig:model6-polynomial+dataset}
		\end{minipage}
	\end{figure}
	
	\begin{figure}[H]
		\begin{minipage}[t]{.5\textwidth}
			\centering
			\includegraphics[width=0.85\linewidth]{../src/img/loss/model7-loss.jpg}
			\captionof{figure}{model7-loss}
			\label{fig:model7-loss}
		\end{minipage}
		~
		\begin{minipage}[t]{.5\textwidth}
			\centering
			\includegraphics[width=1\linewidth]{../src/img/model7-polynomial+dataset.png}
			\captionof{figure}{model7-polynomial+dataset}
			\label{fig:model7-polynomial+dataset}
		\end{minipage}
	\end{figure}
	
	
	\section{Higher-degree Polynomial}
	The last experiment consist in reducing the training dataset to $10$ 
	observations, and compare fitting a polynomial of degree three with fitting 
	a polynomial of degree four.
	The resulting polynomials are visualised in Figure 
	\ref*{fig:model8-loss}.
	
	\begin{figure}[H]
		\begin{minipage}[t]{.5\textwidth}
			\centering
			\includegraphics[width=0.85\linewidth]{../src/img/loss/model8-loss.jpg}
			\captionof{figure}{model8-loss}
			\label{fig:model8-loss}
		\end{minipage}
		~
		\begin{minipage}[t]{.5\textwidth}
			\centering
			\includegraphics[width=1\linewidth]{../src/img/model8-polynomial+dataset.png}
			\captionof{figure}{model8-polynomial+dataset}
			\label{fig:model8-polynomial+dataset}
		\end{minipage}
	\end{figure}
	For what concerns the loss, the validation loss continues to grow in the 
	first $400$ iterations and after that starts going down, while the 
	training loss is constantly decreasing.
	
	In Figure \ref*{fig:model8-polynomial+dataset} is clear that the estimated 
	polynomial of $4\mathrm{th}$ degree is able to fit well the data in 
	the interval $[-1,2]$ while the polynomial of $3\mathrm{rd}$ degree 
	in the interval $[-1.5,1]$.\\
	
	Increasing the number of observations in the training set up to $8000$, it 
	is possible to have a good understanding of the validation loss behaviour 
	over time. 
	
	\begin{figure}[H]
		\begin{minipage}[t]{.5\textwidth}
			\centering
			\includegraphics[width=0.85\linewidth]{../src/img/loss/model8-loss-test.jpg}
			\captionof{figure}{model8-loss-test}
			\label{fig:model8-loss-test}
		\end{minipage}
		~
		\begin{minipage}[t]{.5\textwidth}
			\centering
			\includegraphics[width=1\linewidth]{../src/img/model8-test-polynomial+dataset.png}
			\captionof{figure}{model8-test-polynomial+dataset}
			\label{fig:model8-test-polynomial+dataset}
		\end{minipage}
	\end{figure}

\end{document}
